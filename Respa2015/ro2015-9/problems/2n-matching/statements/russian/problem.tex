\begin{problem}{Наурыз Cup 2015}{E.in}{E.out}{0.5 секунд}{256 мегабайт}

Скоро состоится командное соревнование <<Наурыз Cup 2015>>. Команда должна состоять ровно из двух участников. Аманчик сильно хочет в нем участвовать. Он достал список всех $2 \cdot N$ ($1 \le N \le 10^5$) участников включая \textbf{себя}. У каждого участника есть свой рейтинг. Рейтинг команды это средний рейтинг двух участников. Чем выше рейтинг команды тем выше его место. Команда занимает место под номером $K+1$, если есть ровно $K$ команд, рейтинг которых \textbf{строго больше}. 

Из всевозможных разбиений, какое самое высокое и самое низкое место может занять команда Аманчика. Аманчик участник под номером 1. 

\InputFile
Первая строка входных данных содержит целое число $N$. Следующая строка содержит $2 \cdot N$ целых чисел $1 \le a_i \le 10^5$, $1 \le i \le 2 \cdot N$, разделенных пробелами. 

\OutputFile
Выведите два числа самое высокое и самое низкое место.

\Examples

\begin{example}
\exmp{3
999 3 1 2 1000 1
}{1 2
}%
\exmp{1
1540 1433
}{1 1
}%
\exmp{3
100000 100000 100000 100000 100000 100000
}{1 1
}%
\end{example}


В первом примере если мы разобьем участников следующим образом (999, 2) (3, 1) (1000, 1) то команда Аманчика (999, 2) и команда (1000, 1) возьмут первые места, а команда (3, 1) возьмет третье место. А если мы разобьем следующим образом (999, 1) (1000, 2) (3, 1) то команда Аманчика возьмет второе место. Из всевозможных разбиений, указанные выше будут соответствовать самым высоким и самым низким местам.    

\Scoring
Данная задача содержит четыре подзадачи:
\begin{enumerate}
\item $1 \le N \le 3$. Оценивается в $7$ баллов.
\item $1 \le N \le 6$. Оценивается в $19$ баллов.
\item $1 \le N \le 2500$. Оценивается в $31$ балл.
\item $1 \le N \le 10^5$. Оценивается в $43$ балла.
\end{enumerate}

Каждая следующая подзадача оценивается только при прохождении всех предыдущих.

\end{problem}
