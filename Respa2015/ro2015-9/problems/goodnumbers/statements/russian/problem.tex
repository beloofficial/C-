\begin{problem}{Помощь Жандосу}{D.in}{D.out}{0.5 секунд}{256 мегабайт}

На прошлой неделе на уроке математики Жандос узнал как взять число по модулю. Определим A по модулю B как остаток от деления A на B и обозначим его как A mod B. Учитель по математике задал домашнее задание: найти количество чисел В от L до R включительно, что A mod B = C.

Жандосу нужно сделать домашнюю работу, но он хочет посмотреть футбол. Потому он обратился к вам за помощью. Напишите программу, которая по числам A, C, L, R найдет количество чисел B от L до R включительно, что A mod B = C.


\InputFile
В единственной строке ввода даны целые числа A, C, L, R ($0 \le A, C, L, R \le 10^9$).


\OutputFile
Выведите ответ на задачу.

\Examples

\begin{example}
\exmp{21 5 1 21
}{2
}%
\exmp{32443 463 457 3716
}{12
}%
\end{example}

\Note
\Scoring

Баллы выдаются пропорционально количеству пройденных тестов.

Не менее 25\% тестов с ограничениями $0 \le R - L \le 10^6$.

\end{problem}
