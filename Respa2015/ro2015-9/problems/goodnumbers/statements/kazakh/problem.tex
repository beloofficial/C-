\begin{problem}{Жандосқа көмек}{D.in}{D.out}{0.5 seconds}{256 megabytes}

Өткен аптадағы математика сабағында Жандос санды модуль бойынша алуды үйренді. A санын B саны бойынша модуль алуды A санын B санына бөлгендегі қалдық деп анықтаймыз және A mod B түрінде белгілейміз. Математика мұғалімі A mod B = C болатын L санынан R (қосып алынғанда) санына дейінгі бүкіл B сандарының санын табуды үй тапсырмасы ретінде берді.

Жандосқа үй жұмысын жасау керек, алайда ол оның орнына футбол қарағысы келеді.
Сондықтан ол сізден көмек беруіңізді сұранды. Берілген A, C, L, R сандары бойынша L санынан R (қосып алынғанда) санына дейінгі A mod B = C болатын барлық B сандарының санын табатын бағдарлама жазыңыз.


\InputFile
Енгізу файлының жалғыз қатарында A, C, L, R ($0 \le A, C, L, R \le 10^9$) бүтін сандары берілген.


\OutputFile
Есептің жауабын шығарыңыз.

\Examples

\begin{example}
\exmp{21 5 1 21
}{2
}%
\exmp{32443 463 457 3716
}{12
}%
\end{example}

\Note
\Scoring

Бұл есепте өткен тесттердің санына қарай пропорционалды бағаланады.

Кемдегенде 25\% тесттерінде $0 \le R - L \le 10^6$.


\end{problem}
