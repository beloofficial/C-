\begin{problem}{Пальмаларды көшіру}{F.in}{F.out}{0.5 секунд}{256 мегабайт}

Соңғы кездері Нағыз Университетіндегі (НУ) пальмалар көптеген ыңғайсыздықтар туғызуда. Бұл ағаштардың биіктігін қадағалау ең тәжірибелі деген бағбандардың өзіне де оңай іс болмауда. Ғимараттың сыртқы келбетін жақсарту мақсатында ағаштардың биіктіктерінің тізбегі кемімейтін ретте болатындай кейбір пальмаларды көшіру шешімі қабылданды. 

Ағаштарды көшіру іс-шарасы кез келген ретте орындала береді, және де кейбір ағаштарды бірнеше рет көшіре беруге де болады. Оған қоса осында жұмыс істеп жүрген тәжірибелі бағбандар пальманы басқа екі пальманың арасына, тізбектің басына немесе соңына көшіре алады.

Өкінішке орай, ағашты көшіру ең мықты деген ағаштың өзіне де кері әсерін тигізеді. Сондықтан да бұл іс-шарадағы ұстанатын ең басты қағида - бір рет болсын көшірілген пальмалардың санын барынша азайту.

Тәжірибелі бағбандар әрбір пальманың ешбір басқа пальмаға ұқсамайтынын біледі, және де олардың әрқайсысының көшіруге кететін құны белгілі. Сондықтан да олар жұмысты бастардың алдын соңында барлық ағаштар дұрыс ретпен тұратындай және де көшірілген ағаштардың саны минимум болатындай жалпы жұмысқа кететін минимум шығынды білгісі келеді. Осыны есептеп бере аласыз ба?

\InputFile
Бірінші қатарда бүтін $N$ ($1 \le N \le 10^5$) саны --- университет атриумындағы пальмалар саны. 

Екінші қатарда бос орындармен бөлінген $N$ бүтін сан $H_1,H_2,\dots,H_N$ ($1 \le H_i \le 10^9$) --- пальмалардың биіктіктері, бастапқы өсіп тұрғандағы ретімен берілген.

Екінші қатарда бос орындармен бөлінген $N$ бүтін сан $C_1,C_2,\dots,C_N$ ($1 \le C_i \le 10^9$) --- пальмаларды көшірудің құны, дәл сол ретпен берілген.

\OutputFile
Соңында барлық ағаштардың биіктіктері кемімейтін ретпен тұратындай, және де көшірілген ағаштардың саны минимум болатындай жалпы жұмысқа кететін минимум шығынды шығарыңыз. 

\Examples

\begin{example}
\exmp{3
2 1 3
5 6 2
}{5
}%
\end{example}


Бірінші мысалда бір ғана пальманы көшіру жеткілікті --- не біріншісін (биіктігі 2), не екіншісін (биіктігі 1). Бірінші пальманы көшіруге кететін шығын аз болғандықтан, сол пальма көшіріледі (жалпы жұмыс барысы: биіктігі 2 болатын пальма биіктігі 1 және 3 болатын пальмалардың арасына көшіріледі).

\Scoring
Берілген тапсырма үш бөліктен тұрады:
\begin{enumerate}
\item $1 \le N \le 20$. Бағалануы $25$ ұпай.
\item $1 \le N \le 1000$. Бағалануы $25$ ұпай.
\item $1 \le N \le 10^5$. Бағалануы $50$ ұпай.
\end{enumerate}

Әр бөлік өзінен алдынғы бөліктер орындалғанда ғана бағаланады.

\end{problem}
