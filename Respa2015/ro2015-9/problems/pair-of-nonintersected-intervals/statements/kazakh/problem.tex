\begin{problem}{Temirulan vs Pernekhan}{C.in}{C.out}{0.5 секунд}{256 мегабайт}

Темірұлан мен Пернеханға $1 \le N \le 5000$ бүтін оң сандардан тұратын $A$ тізбегі сыйланды. Әрқайысысы тізбектен қандай да бір бос емес қатарлас сандар тізбегін алуы керек. Темірұланның тізбегі Пернеханның тізбегінен ертерек басталады және олардың таңдаған тізбектерінде бірдей сан болмауы керек. Қасында бақылап отырған Айдос, олар бұл тізбекті қанша түрлі жолмен өзара бөлісетініне қатты қызығып кетті. Осы тапсырмаға бағдарлама жазып, Айдосқа көмектесіңіз.

\InputFile
Бірінші қатарда бүтін $N$ саны беріледі. Келесі қатарда бос орындар арқылы $N$ бүтін сан беріледі. $1 \le A_i \le N$, $1 \le i \le N$.

\OutputFile
Тапсырманың жауабы болатын жалғыз санды шығарыңыз.

\Examples

\begin{example}
\exmp{3
1 2 3
}{5
}%
\exmp{4
1 2 3 2
}{9
}%
\exmp{1
1
}{0
}%
\end{example}


Екінші тесттік мысалда төменде көрсетілгендей бөлісе алады:

\{ [1] [2] 3 2 \}, \{ [1] [2 3] 2 \}, \{ [1] [2 3 2] \},

\{ [1] 2 [3] 2 \}, \{ [1] 2 [3 2] \}, \{ [1] 2 3 [2] \}, 

\{ [1 2] [3] 2 \}, \{ 1 [2] [3] 2 \}, \{ 1 2 [3] [2] \}

\Scoring
Бұл тапсырма төрт бөлікке бөлінген.
\begin{enumerate}
\item $1 \le N \le 50$. $11$ ұпай.
\item $1 \le N \le 500$. $21$ ұпай.
\item $1 \le N \le 2000$. $31$ ұпай.
\item $1 \le N \le 5000$. $37$ ұпай.
\end{enumerate}

Әр бөлік өзінен алдынғы бөліктер орындалғанда ғана бағаланады.

\end{problem}
